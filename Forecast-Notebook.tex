\PassOptionsToPackage{unicode=true}{hyperref} % options for packages loaded elsewhere
\PassOptionsToPackage{hyphens}{url}
%
\documentclass[]{article}
\usepackage{lmodern}
\usepackage{amssymb,amsmath}
\usepackage{ifxetex,ifluatex}
\usepackage{fixltx2e} % provides \textsubscript
\ifnum 0\ifxetex 1\fi\ifluatex 1\fi=0 % if pdftex
  \usepackage[T1]{fontenc}
  \usepackage[utf8]{inputenc}
  \usepackage{textcomp} % provides euro and other symbols
\else % if luatex or xelatex
  \usepackage{unicode-math}
  \defaultfontfeatures{Ligatures=TeX,Scale=MatchLowercase}
\fi
% use upquote if available, for straight quotes in verbatim environments
\IfFileExists{upquote.sty}{\usepackage{upquote}}{}
% use microtype if available
\IfFileExists{microtype.sty}{%
\usepackage[]{microtype}
\UseMicrotypeSet[protrusion]{basicmath} % disable protrusion for tt fonts
}{}
\IfFileExists{parskip.sty}{%
\usepackage{parskip}
}{% else
\setlength{\parindent}{0pt}
\setlength{\parskip}{6pt plus 2pt minus 1pt}
}
\usepackage{hyperref}
\hypersetup{
            pdftitle={Untitled},
            pdfauthor={Benjamin Anderson},
            pdfborder={0 0 0},
            breaklinks=true}
\urlstyle{same}  % don't use monospace font for urls
\usepackage[margin=1in]{geometry}
\usepackage{color}
\usepackage{fancyvrb}
\newcommand{\VerbBar}{|}
\newcommand{\VERB}{\Verb[commandchars=\\\{\}]}
\DefineVerbatimEnvironment{Highlighting}{Verbatim}{commandchars=\\\{\}}
% Add ',fontsize=\small' for more characters per line
\usepackage{framed}
\definecolor{shadecolor}{RGB}{248,248,248}
\newenvironment{Shaded}{\begin{snugshade}}{\end{snugshade}}
\newcommand{\AlertTok}[1]{\textcolor[rgb]{0.94,0.16,0.16}{#1}}
\newcommand{\AnnotationTok}[1]{\textcolor[rgb]{0.56,0.35,0.01}{\textbf{\textit{#1}}}}
\newcommand{\AttributeTok}[1]{\textcolor[rgb]{0.77,0.63,0.00}{#1}}
\newcommand{\BaseNTok}[1]{\textcolor[rgb]{0.00,0.00,0.81}{#1}}
\newcommand{\BuiltInTok}[1]{#1}
\newcommand{\CharTok}[1]{\textcolor[rgb]{0.31,0.60,0.02}{#1}}
\newcommand{\CommentTok}[1]{\textcolor[rgb]{0.56,0.35,0.01}{\textit{#1}}}
\newcommand{\CommentVarTok}[1]{\textcolor[rgb]{0.56,0.35,0.01}{\textbf{\textit{#1}}}}
\newcommand{\ConstantTok}[1]{\textcolor[rgb]{0.00,0.00,0.00}{#1}}
\newcommand{\ControlFlowTok}[1]{\textcolor[rgb]{0.13,0.29,0.53}{\textbf{#1}}}
\newcommand{\DataTypeTok}[1]{\textcolor[rgb]{0.13,0.29,0.53}{#1}}
\newcommand{\DecValTok}[1]{\textcolor[rgb]{0.00,0.00,0.81}{#1}}
\newcommand{\DocumentationTok}[1]{\textcolor[rgb]{0.56,0.35,0.01}{\textbf{\textit{#1}}}}
\newcommand{\ErrorTok}[1]{\textcolor[rgb]{0.64,0.00,0.00}{\textbf{#1}}}
\newcommand{\ExtensionTok}[1]{#1}
\newcommand{\FloatTok}[1]{\textcolor[rgb]{0.00,0.00,0.81}{#1}}
\newcommand{\FunctionTok}[1]{\textcolor[rgb]{0.00,0.00,0.00}{#1}}
\newcommand{\ImportTok}[1]{#1}
\newcommand{\InformationTok}[1]{\textcolor[rgb]{0.56,0.35,0.01}{\textbf{\textit{#1}}}}
\newcommand{\KeywordTok}[1]{\textcolor[rgb]{0.13,0.29,0.53}{\textbf{#1}}}
\newcommand{\NormalTok}[1]{#1}
\newcommand{\OperatorTok}[1]{\textcolor[rgb]{0.81,0.36,0.00}{\textbf{#1}}}
\newcommand{\OtherTok}[1]{\textcolor[rgb]{0.56,0.35,0.01}{#1}}
\newcommand{\PreprocessorTok}[1]{\textcolor[rgb]{0.56,0.35,0.01}{\textit{#1}}}
\newcommand{\RegionMarkerTok}[1]{#1}
\newcommand{\SpecialCharTok}[1]{\textcolor[rgb]{0.00,0.00,0.00}{#1}}
\newcommand{\SpecialStringTok}[1]{\textcolor[rgb]{0.31,0.60,0.02}{#1}}
\newcommand{\StringTok}[1]{\textcolor[rgb]{0.31,0.60,0.02}{#1}}
\newcommand{\VariableTok}[1]{\textcolor[rgb]{0.00,0.00,0.00}{#1}}
\newcommand{\VerbatimStringTok}[1]{\textcolor[rgb]{0.31,0.60,0.02}{#1}}
\newcommand{\WarningTok}[1]{\textcolor[rgb]{0.56,0.35,0.01}{\textbf{\textit{#1}}}}
\usepackage{graphicx,grffile}
\makeatletter
\def\maxwidth{\ifdim\Gin@nat@width>\linewidth\linewidth\else\Gin@nat@width\fi}
\def\maxheight{\ifdim\Gin@nat@height>\textheight\textheight\else\Gin@nat@height\fi}
\makeatother
% Scale images if necessary, so that they will not overflow the page
% margins by default, and it is still possible to overwrite the defaults
% using explicit options in \includegraphics[width, height, ...]{}
\setkeys{Gin}{width=\maxwidth,height=\maxheight,keepaspectratio}
\setlength{\emergencystretch}{3em}  % prevent overfull lines
\providecommand{\tightlist}{%
  \setlength{\itemsep}{0pt}\setlength{\parskip}{0pt}}
\setcounter{secnumdepth}{0}
% Redefines (sub)paragraphs to behave more like sections
\ifx\paragraph\undefined\else
\let\oldparagraph\paragraph
\renewcommand{\paragraph}[1]{\oldparagraph{#1}\mbox{}}
\fi
\ifx\subparagraph\undefined\else
\let\oldsubparagraph\subparagraph
\renewcommand{\subparagraph}[1]{\oldsubparagraph{#1}\mbox{}}
\fi

% set default figure placement to htbp
\makeatletter
\def\fps@figure{htbp}
\makeatother


\title{Untitled}
\author{Benjamin Anderson}
\date{November 23, 2020}

\begin{document}
\maketitle

\hypertarget{forecasting-using-xgboost}{%
\section{Forecasting using XGBoost}\label{forecasting-using-xgboost}}

\hypertarget{read-in-the-necessary-packages}{%
\subsubsection{Read in the necessary
packages}\label{read-in-the-necessary-packages}}

\begin{Shaded}
\begin{Highlighting}[]
\KeywordTok{library}\NormalTok{(readr)}
\end{Highlighting}
\end{Shaded}

\begin{verbatim}
## Warning: package 'readr' was built under R version 3.6.3
\end{verbatim}

\begin{Shaded}
\begin{Highlighting}[]
\KeywordTok{library}\NormalTok{(DataExplorer)}
\end{Highlighting}
\end{Shaded}

\begin{verbatim}
## Warning: package 'DataExplorer' was built under R version 3.6.3
\end{verbatim}

\begin{Shaded}
\begin{Highlighting}[]
\KeywordTok{library}\NormalTok{(tidyverse)}
\end{Highlighting}
\end{Shaded}

\begin{verbatim}
## Warning: package 'tidyverse' was built under R version 3.6.2
\end{verbatim}

\begin{verbatim}
## -- Attaching packages --------------------------------------- tidyverse 1.3.0 --
\end{verbatim}

\begin{verbatim}
## v ggplot2 3.3.2     v dplyr   1.0.2
## v tibble  3.0.4     v stringr 1.4.0
## v tidyr   1.1.2     v forcats 0.5.0
## v purrr   0.3.4
\end{verbatim}

\begin{verbatim}
## Warning: package 'ggplot2' was built under R version 3.6.3
\end{verbatim}

\begin{verbatim}
## Warning: package 'tibble' was built under R version 3.6.3
\end{verbatim}

\begin{verbatim}
## Warning: package 'tidyr' was built under R version 3.6.3
\end{verbatim}

\begin{verbatim}
## Warning: package 'purrr' was built under R version 3.6.3
\end{verbatim}

\begin{verbatim}
## Warning: package 'dplyr' was built under R version 3.6.3
\end{verbatim}

\begin{verbatim}
## Warning: package 'forcats' was built under R version 3.6.3
\end{verbatim}

\begin{verbatim}
## -- Conflicts ------------------------------------------ tidyverse_conflicts() --
## x dplyr::filter() masks stats::filter()
## x dplyr::lag()    masks stats::lag()
\end{verbatim}

\begin{Shaded}
\begin{Highlighting}[]
\KeywordTok{library}\NormalTok{(lubridate)}
\end{Highlighting}
\end{Shaded}

\begin{verbatim}
## Warning: package 'lubridate' was built under R version 3.6.3
\end{verbatim}

\begin{verbatim}
## 
## Attaching package: 'lubridate'
\end{verbatim}

\begin{verbatim}
## The following objects are masked from 'package:base':
## 
##     date, intersect, setdiff, union
\end{verbatim}

\begin{Shaded}
\begin{Highlighting}[]
\KeywordTok{library}\NormalTok{(dplyr)}
\KeywordTok{library}\NormalTok{(caret)}
\end{Highlighting}
\end{Shaded}

\begin{verbatim}
## Warning: package 'caret' was built under R version 3.6.3
\end{verbatim}

\begin{verbatim}
## Loading required package: lattice
\end{verbatim}

\begin{verbatim}
## 
## Attaching package: 'caret'
\end{verbatim}

\begin{verbatim}
## The following object is masked from 'package:purrr':
## 
##     lift
\end{verbatim}

\begin{Shaded}
\begin{Highlighting}[]
\KeywordTok{library}\NormalTok{(xgboost)}
\end{Highlighting}
\end{Shaded}

\begin{verbatim}
## Warning: package 'xgboost' was built under R version 3.6.3
\end{verbatim}

\begin{verbatim}
## 
## Attaching package: 'xgboost'
\end{verbatim}

\begin{verbatim}
## The following object is masked from 'package:dplyr':
## 
##     slice
\end{verbatim}

\begin{Shaded}
\begin{Highlighting}[]
\KeywordTok{library}\NormalTok{(plyr) }
\end{Highlighting}
\end{Shaded}

\begin{verbatim}
## Warning: package 'plyr' was built under R version 3.6.3
\end{verbatim}

\begin{verbatim}
## ------------------------------------------------------------------------------
\end{verbatim}

\begin{verbatim}
## You have loaded plyr after dplyr - this is likely to cause problems.
## If you need functions from both plyr and dplyr, please load plyr first, then dplyr:
## library(plyr); library(dplyr)
\end{verbatim}

\begin{verbatim}
## ------------------------------------------------------------------------------
\end{verbatim}

\begin{verbatim}
## 
## Attaching package: 'plyr'
\end{verbatim}

\begin{verbatim}
## The following objects are masked from 'package:dplyr':
## 
##     arrange, count, desc, failwith, id, mutate, rename, summarise,
##     summarize
\end{verbatim}

\begin{verbatim}
## The following object is masked from 'package:purrr':
## 
##     compact
\end{verbatim}

\begin{Shaded}
\begin{Highlighting}[]
\KeywordTok{library}\NormalTok{(tidyverse)}
\KeywordTok{library}\NormalTok{(data.table)}
\end{Highlighting}
\end{Shaded}

\begin{verbatim}
## Warning: package 'data.table' was built under R version 3.6.3
\end{verbatim}

\begin{verbatim}
## 
## Attaching package: 'data.table'
\end{verbatim}

\begin{verbatim}
## The following objects are masked from 'package:lubridate':
## 
##     hour, isoweek, mday, minute, month, quarter, second, wday, week,
##     yday, year
\end{verbatim}

\begin{verbatim}
## The following objects are masked from 'package:dplyr':
## 
##     between, first, last
\end{verbatim}

\begin{verbatim}
## The following object is masked from 'package:purrr':
## 
##     transpose
\end{verbatim}

\begin{Shaded}
\begin{Highlighting}[]
\KeywordTok{library}\NormalTok{(Matrix)}
\end{Highlighting}
\end{Shaded}

\begin{verbatim}
## 
## Attaching package: 'Matrix'
\end{verbatim}

\begin{verbatim}
## The following objects are masked from 'package:tidyr':
## 
##     expand, pack, unpack
\end{verbatim}

\begin{Shaded}
\begin{Highlighting}[]
\KeywordTok{library}\NormalTok{(psych)}
\end{Highlighting}
\end{Shaded}

\begin{verbatim}
## Warning: package 'psych' was built under R version 3.6.3
\end{verbatim}

\begin{verbatim}
## 
## Attaching package: 'psych'
\end{verbatim}

\begin{verbatim}
## The following objects are masked from 'package:ggplot2':
## 
##     %+%, alpha
\end{verbatim}

\begin{Shaded}
\begin{Highlighting}[]
\KeywordTok{library}\NormalTok{(gbm)}
\end{Highlighting}
\end{Shaded}

\begin{verbatim}
## Warning: package 'gbm' was built under R version 3.6.3
\end{verbatim}

\begin{verbatim}
## Loaded gbm 2.1.8
\end{verbatim}

\hypertarget{read-in-the-datasets}{%
\subsubsection{Read in the datasets}\label{read-in-the-datasets}}

\begin{Shaded}
\begin{Highlighting}[]
\NormalTok{store.train <-}\StringTok{ }\KeywordTok{read_csv}\NormalTok{(}\StringTok{'train.csv'}\NormalTok{)}
\end{Highlighting}
\end{Shaded}

\begin{verbatim}
## 
## -- Column specification --------------------------------------------------------
## cols(
##   date = col_date(format = ""),
##   store = col_double(),
##   item = col_double(),
##   sales = col_double()
## )
\end{verbatim}

\begin{Shaded}
\begin{Highlighting}[]
\NormalTok{store.test <-}\StringTok{ }\KeywordTok{read_csv}\NormalTok{(}\StringTok{'test.csv'}\NormalTok{)}
\end{Highlighting}
\end{Shaded}

\begin{verbatim}
## 
## -- Column specification --------------------------------------------------------
## cols(
##   id = col_double(),
##   date = col_date(format = ""),
##   store = col_double(),
##   item = col_double()
## )
\end{verbatim}

\begin{Shaded}
\begin{Highlighting}[]
\NormalTok{store <-}\StringTok{ }\KeywordTok{bind_rows}\NormalTok{(store.train, }
\NormalTok{                   store.test, }
                   \DataTypeTok{.id =} \StringTok{'me'}\NormalTok{) }\OperatorTok\StringTok{ }
\StringTok{  }\KeywordTok{select}\NormalTok{(}\OperatorTok{-}\NormalTok{me)}
\end{Highlighting}
\end{Shaded}

\hypertarget{pull-out-weekdays-months-and-year}{%
\subsubsection{Pull out weekdays, months and
year}\label{pull-out-weekdays-months-and-year}}

\begin{Shaded}
\begin{Highlighting}[]
\NormalTok{store}\OperatorTok{$}\NormalTok{dayofweek <-}\StringTok{ }\NormalTok{store}\OperatorTok{$}\NormalTok{date }\OperatorTok\StringTok{ }\KeywordTok{weekdays}\NormalTok{()}
\NormalTok{store}\OperatorTok{$}\NormalTok{month <-}\StringTok{ }\NormalTok{store}\OperatorTok{$}\NormalTok{date }\OperatorTok\StringTok{ }\KeywordTok{month}\NormalTok{()}
\NormalTok{store}\OperatorTok{$}\NormalTok{year <-}\StringTok{ }\NormalTok{store}\OperatorTok{$}\NormalTok{date }\OperatorTok\StringTok{ }\KeywordTok{year}\NormalTok{()}
\NormalTok{store <-}\StringTok{ }\NormalTok{store }\OperatorTok\StringTok{ }\KeywordTok{select}\NormalTok{(}\OperatorTok{-}\NormalTok{date)}
\end{Highlighting}
\end{Shaded}

\hypertarget{split-up-the-data}{%
\subsubsection{Split up the data}\label{split-up-the-data}}

\begin{Shaded}
\begin{Highlighting}[]
\NormalTok{store.train <-}\StringTok{ }\NormalTok{store }\OperatorTok\StringTok{ }\KeywordTok{filter}\NormalTok{(}\OperatorTok{!}\KeywordTok{is.na}\NormalTok{(sales))}
\NormalTok{store.test <-}\StringTok{ }\NormalTok{store }\OperatorTok\StringTok{ }\KeywordTok{filter}\NormalTok{(}\KeywordTok{is.na}\NormalTok{(sales))}
\end{Highlighting}
\end{Shaded}

\hypertarget{the-data-is-clean-and-has-no-missing-variables}{%
\subsubsection{The data is clean and has no missing
variables}\label{the-data-is-clean-and-has-no-missing-variables}}

\begin{Shaded}
\begin{Highlighting}[]
\KeywordTok{plot_missing}\NormalTok{(store.train)}
\end{Highlighting}
\end{Shaded}

\includegraphics{Forecast-Notebook_files/figure-latex/unnamed-chunk-5-1.pdf}

\hypertarget{predictions}{%
\subsection{Predictions}\label{predictions}}

\hypertarget{set-up-training-parameters-and-tuning-grid}{%
\subsubsection{Set up training parameters and tuning
grid}\label{set-up-training-parameters-and-tuning-grid}}

\begin{Shaded}
\begin{Highlighting}[]
\NormalTok{control <-}\StringTok{ }\KeywordTok{trainControl}\NormalTok{(}\DataTypeTok{method=}\StringTok{'repeatedcv'}\NormalTok{,}
                        \DataTypeTok{number=}\DecValTok{3}\NormalTok{,}
                        \DataTypeTok{repeats=}\DecValTok{2}\NormalTok{)}

\NormalTok{grid_default <-}\StringTok{ }\KeywordTok{expand.grid}\NormalTok{(}\DataTypeTok{nrounds =} \DecValTok{250}\NormalTok{,}
                            \DataTypeTok{max_depth =} \DecValTok{10}\NormalTok{,}
                            \DataTypeTok{eta =} \FloatTok{0.3}\NormalTok{,}
                            \DataTypeTok{gamma =} \DecValTok{15}\OperatorTok{:}\DecValTok{20}\NormalTok{,}
                            \DataTypeTok{colsample_bytree =} \FloatTok{.5}\NormalTok{,}
                            \DataTypeTok{min_child_weight =} \DecValTok{25}\NormalTok{,}
                            \DataTypeTok{subsample =} \DecValTok{1}\OperatorTok{:}\DecValTok{2}\NormalTok{)}
\end{Highlighting}
\end{Shaded}

\hypertarget{begin-the-model-training-using-the-caret-function}{%
\subsubsection{Begin the model training using the `caret'
function}\label{begin-the-model-training-using-the-caret-function}}

\begin{Shaded}
\begin{Highlighting}[]
\CommentTok{# metric <- "Accuracy"}
\CommentTok{# set.seed(123)}
\CommentTok{# xgb_default <- train(sales ~ .,}
\CommentTok{#                     data=store.train %>% select(-id),}
\CommentTok{#                     method='xgbTree',}
\CommentTok{#                     trControl=control,}
\CommentTok{#                     tuneGrid = grid_default,}
\CommentTok{#                     objective = "reg:squarederror")}
\CommentTok{# }
\CommentTok{# # Save the gradient boosted model}
\CommentTok{# saveRDS(xgb_default, 'xgb_default.RDS')}

\CommentTok{# Read in the model if it has already run}
\NormalTok{xgb_default <-}\StringTok{ }\KeywordTok{readRDS}\NormalTok{(}\StringTok{'xgb_default.RDS'}\NormalTok{)}
\end{Highlighting}
\end{Shaded}

\hypertarget{information-regarding-the-model}{%
\subsubsection{Information regarding the
model}\label{information-regarding-the-model}}

\begin{Shaded}
\begin{Highlighting}[]
\KeywordTok{names}\NormalTok{(xgb_default)}
\end{Highlighting}
\end{Shaded}

\begin{verbatim}
##  [1] "method"       "modelInfo"    "modelType"    "results"      "pred"        
##  [6] "bestTune"     "call"         "dots"         "metric"       "control"     
## [11] "finalModel"   "preProcess"   "trainingData" "resample"     "resampledCM" 
## [16] "perfNames"    "maximize"     "yLimits"      "times"        "levels"      
## [21] "terms"        "coefnames"    "contrasts"    "xlevels"
\end{verbatim}

\begin{Shaded}
\begin{Highlighting}[]
\KeywordTok{plot}\NormalTok{(xgb_default)}
\end{Highlighting}
\end{Shaded}

\includegraphics{Forecast-Notebook_files/figure-latex/unnamed-chunk-8-1.pdf}

\begin{Shaded}
\begin{Highlighting}[]
\NormalTok{xgb_default}\OperatorTok{$}\NormalTok{bestTune}
\end{Highlighting}
\end{Shaded}

\begin{verbatim}
##   nrounds max_depth eta gamma colsample_bytree min_child_weight subsample
## 5     250        10 0.3    17              0.5               25         1
\end{verbatim}

\hypertarget{make-a-prediction-and-write-out-the-.csv-file}{%
\subsubsection{Make a prediction and write out the .csv
file}\label{make-a-prediction-and-write-out-the-.csv-file}}

\begin{Shaded}
\begin{Highlighting}[]
\NormalTok{predictions <-}\StringTok{ }\KeywordTok{data.frame}\NormalTok{(}\DataTypeTok{id=}\NormalTok{store.test}\OperatorTok{$}\NormalTok{id, }
                          \DataTypeTok{sales =}\NormalTok{ (}\KeywordTok{predict}\NormalTok{(xgb_default, }
                                                             \DataTypeTok{newdata=}\NormalTok{store.test)))}

\CommentTok{# write to a csv}
\KeywordTok{write.csv}\NormalTok{(predictions,}
          \StringTok{"submission.csv"}\NormalTok{, }
          \DataTypeTok{row.names =} \OtherTok{FALSE}\NormalTok{)}
\end{Highlighting}
\end{Shaded}

The model is accurate and does relatively well compared to other models
on Kaggle with a SMAPE score of 14.3.

\end{document}
